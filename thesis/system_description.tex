\chapter{System Description}\label{chapter:system}

\section{System Overview}

TODO: Rework overview

As mentioned before, the goal of thesis is the design and implementation of a system that allows the execution of SQL queries that join a real-time data stream and an external dataset. The prerequisites for the system are extensibility, scalablitily, fault tolerance and most importantly enabling the execution of low latency SQL queries.

----------------------
join performance
an arbitrary time window of the data stream.
increasing the query latency
multiple

Performing an SQL join combines records from two tables. The join effectively creates a third table which combines the information from both of them. This is at some expense in terms of the time it takes to compute the join. 

While it is also possible to simply maintain a denormalized table if speed is important, duplicate information may take extra space, and add the expense and complexity of maintaining data integrity if data which is duplicated later changes.

Denormalization is the process of attempting to optimize the read performance of a database by adding redundant data or by grouping data. In some cases, denormalization is a means of addressing performance or scalability in relational database software. Denormalization introduces a trade-off, speeding up reads while slowing down writes 

move the join cost to the Storm processing
pre-join
-----------------------

From a high level, the system implemented for our IXP network data usecase consists of 4 major parts that can be seen in Figure \ref{figure:system_overview}. In the first part, the network data is generated by the switches of an IXP and collected by a host running a Kafka producer. There, the useful fields are extracted from the captured packets and published to the Kafka topic. The second component of the system is the Kafka topic that stores the data stream at the Kafka cluster. In the next part, the data stream is processed by a Storm topology. The topology contains the IP to AS Bolt, that performs the join of the data stream and the AS dataset in-memory, since the size of the dataset is small enough. It also contains the IP to DNS Bolt, that performs the join of the data stream and the Reverse DNS dataset using Get operations on the HBase table where the dataset is stored, since it does not fit in the bolt's memory. Finally, in the last part the denormalized network data is stored at a Phoenix table in HBase, allowing Phoenix clients to perform low latency SQL queries to it.

\begin{figure}[H]
\centering
\includegraphics[width=\textwidth]{figures/system_overview}
\caption{Storm architecture overview}
\label{figure:system_overview}
\end{figure}

The system's \textbf{scalability} is achieved by using distributed frameworks and technologies for it's implementation. Kafka topics consist of partitions that are distributed over a cluster of Kafka brokers. Storm topologies run over a cluster of Supervisors and multiple instances of any component of the topology (spout or bolt) can run at the same time. The output Phoenix table is stored in HBase and subsequently in the HDFS, which both distributed technologies run on clusters of DataNodes and RegionServers respectively. Moreover, Phoenix can parallelize queries to take full advantage of the HBase cluster.

\textbf{Fault tolerance} is very important for our system since it will be constantly running over extended periods of time, processing real-time data. First of all, Kafka topic partitions can be replicated across multiple Kafka brokers, allowing data input by the Kafka producer and consumption by the Storm topology in the case of a broker failure. Storm topologies are also fault tolerant and in case of a Supervisor failure Nimbus reassigns the tasks as necessary. Storm also keeps track of failed tuples and is able to replay them since Kafka retains a topic's data for a configurable period of time. This allows us to restart Storm topologies without skipping any data. Finally, the output Phoenix table that is stored in HBase is replicated by the underlying HDFS, allowing its data to be available in the case of a DataNode or RegionServer failure.

Using the Storm framework provides \textbf{extensibility} to our system. Extending the functionality of the Storm topology is as simple as adding an extra bolt to the topology. For example, the processing for the join of the data stream with a new external dataset can be added by implementing the new bolt and placing it before the output Phoenix Bolt. We discern two cases with respect to the size of the external dataset. If the dataset's size is small enough, we can load it in the bolt's memory and perform the join in-memory. Otherwise, when the dataset does not fit in memory, we store it in an HBase table and perform the join using Get operations.

In the following Sections of this Chapter we offer a detailed description for all of the system's components.


\section{Data Generation and Input}

\subsection{IXP Switch}

The data stream that is processed by our system is generated by an \emph{sFlow agent} running on a switch that processes traffic in an IXP. sFlow is an industry standard technology for monitoring high speed switched networks and is supported by multiple network device manufacturers. The sFlow agent performs random sampling of packets processed by the switch. By default, the agent samples the first 128 bytes of 1 in every 2048 packets.

The flow samples are sent as \emph{sFlow datagrams} (UDP packets) to the \emph{sFlow collector}, described in Subsection \ref{subsection:system_kafka_producer}. The sFlow collector can accept sFlow datagrams from multiple sFlow agents, allowing us to process a data stream that combines flow samples generated by multiple switches that are used in the same IXP.

\subsection{Kafka Producer}\label{subsection:system_kafka_producer}

The sFlow datagrams are sent by the sFlow agents of the IXP switches to an sFlow collector running at a specified host. This sFlow collector collects the flow samples from all the switches and makes them available for further processing. In our implementation we use \texttt{sflowtool}, a tool functions as an sFlow collector and translates the flow samples to a simple-to-parse ASCII format.

The same host runs a Kafka producer script that preprocesses the flow samples and publishes the useful information to a Kafka topic. This script reads the output of our sFlow collector \texttt{sflowtool} and extracts the following useful fields for each sampled packet:
\begin{itemize}
\item \texttt{sourceIP}: source IP address in dot-decimal notation
\item \texttt{destinationIP}: destination IP address in dot-decimal notation
\item \texttt{protocol}: IP protocol number (6 for TCP, 17 for UDP)
\item \texttt{sourcePort}: source port number
\item \texttt{destinationPort}: destination port number
\item \texttt{ipSize}: total length of the IP packet
\item \texttt{dateTime}: Unix timestamp of the packet's capture time in microseconds. This field is generated by the script while preprocessing each packet. 
\end{itemize}

After the extraction, we compose a message containing the fields in CSV format. The script is running a Kafka producer that publishes these messages to the Kafka topic topic \texttt{netdata} that is stored at the Kafka cluster. 

Algorithm \ref{algorithm:kafka_producer} outlines the script implementation.

\begin{algorithm}[H]
\begin{algorithmic}[1]
\For {line in sFlowToolOutput}
\State fields = line.split(",")
\State sourceIP = fields[9]
\State destinationIP = fields[10]
\State protocol = fields[11]
\State sourcePort = fields[14]
\State destinationPort = fields[15]
\State ipSize = fields[17]
\State dateTime = int(time.time()*1000000)
\State message = "\{\},\{\},\{\},\{\},\{\},\{\},\{\}".format(sourceIP, destinationIP, protocol, sourcePort, destinationPort, ipSize, dateTime)
\State kafkaProducer.send\_messages("netdata", message)
\EndFor
\end{algorithmic}
\caption{Kafka Producer}
\label{algorithm:kafka_producer}
\end{algorithm}

Messages can be sent to a Kafka topic either synchronously or asynchronously. Synchronous send publishes the messages immediately, whereas asynchronous send accumulates them in memory batches  multiple messages in a single request. As we will see in Subsection \ref{subsection:benchmarks_kafka_batch} batching can greatly increase the performance of the producer, therefore we choose to use asynchronous send.


\section{Kafka Topic}

The preprocessed messages containing the useful fields in CSV format are stored at the \texttt{netdata} Kafka topic in the Kafka cluster. For scalability and load balancing we set the number of the topic's partitions equal to the number of the brokers of the Kafka cluster. In this way, the write and read requests of the producer and the consumers respectively are distributed over the cluster.

To provide fault tolerance, we also set a replication factor of 2 for the topic. This means that every partition is replicated and stored in 2 brokers, the leader that handles all the requests for the partition, and the follower that is replicating the leader. In case of failure on the leader, the follower can take over and handle the requests for the partition.

As we mentioned in Section \ref{section:theory_kafka}, all published messages remain stored at the brokers for a configurable period of time, whether or not they have been consumed. This allows the Storm topology to replay previously read messages in case of failure. The default data retention window for the topic is 7 days.


\section{Storm Topology}

The Storm topology is the heart of our system. This is where the processing of the data stream is performed. The topology consists of 1 spout and 4 bolts in a pipeline setup: Kafka Spout, Split Fields Bolt, IP to AS Bolt, IP to DNS Bolt and Phoenix Bolt. In short, the topology reads messages from a Kafka topic, extracts the useful fields from the messages, performs the join of the data stream and the external datasets and finally stores the augmented data in a Phoenix table. The topology has acking enabled, which guarantees that every message from the topic will be processed and will be replayed in case it fails.

The overview of the Storm topology for our network data usecase can be seen in Figure \ref{figure:system_storm_topology}. As we will see in Section \ref{}, the functionality of the topology can be extended by adding more bolts that perform the join of the data stream and another external dataset right before the Phoenix Bolt.

\begin{figure}[H]
\centering
\includegraphics[width=\textwidth]{figures/system_storm_topology}
\caption{Storm topology overview}
\label{figure:system_storm_topology}
\end{figure}

\subsection{Kafka Spout}

The source of data stream in our topology is the Kafka Spout. The spout is a Kafka consumer that reads messages from the \texttt{netdata} Kafka topic and emits them to the Split Fields Bolt. The maximum parallelism of the Kafka spout is the number of the topic's partitions, because any instances of the spout further than that would not read any data.

The Kafka Spout stores the offset of the consumer for each partition of the topic in Zookeeper. In this way, if a failure happens the topology can be restarted and resume reading messages from the last one that was executed successfully by the topology.

\subsection{Split Fields Bolt}

The tuple emitted by the Kafka Spout has a single field: a message from the topic containing the useful fields of the packet in CSV format. The Split Fields Bolt extracts these fields from the message. In addition to that the bolt computes the integer representations of the source and destiantion IP addresses, which are usually more useful than the IP addresses in dot-decimal notation.

After processing the Kafka message, the Split Fileds Bolt emits a tuple containing the following fields: sourceIP, sourceIPInt, destinationIP, destinationIPInt, protocol, sourcePort, destinationPort, ipSize, dateTime. 

\subsection{IP to AS Bolt}

The natural join of the data stream and the Autonomous System dataset is performed by the IP to AS Bolt. The Autonomous System dataset maps IP addresses to AS number and name. The data contained in the dataset are stored in CSV format and have 3 fields: the first IP address contained in the AS, the last IP address contained in the AS and the AS number and name. The IP adresses are stored in their integer representation format. The dataset file must be stored in a location accessible by all the Storm supervisors, such as the HDFS. Further information on the dataset is available in Subsection \ref{subsection:benchmarks_as_dataset}.

The defining characteristic of the Autonomous System dataset is that its size (13 MB) is small enough to fit in the memory, which is the optimal way to perform the join of the stream and the dataset. During the initialization of the topology the \texttt{prepare} method of the IP to AS Bolt is called and loads the dataset in a TreeMap structure. A TreeMap is a map implementation based on red-black trees, a variation of binary search trees. For each record of the dataset we insert 2 records in the TreeMap, containing the start and stop IP address for each AS along with the AS number and name.

The helper method \texttt{ipToAS} takes an IP address in integer representation format as input and returns the name and number of the AS it belongs. More specifically, by using the TreeMap's ceilingKey and get methods we find the first AS boundary IP address larger or equal to the input IP address. If this address is equal to the input IP address or coresponds to the last address of an AS IP address range, then the AS it belongs is the one we are looking for and its number and name is returned by the method. Otherwise the IP address provided does not belong to any AS according to the dataset and the \texttt{ipToAS} method returns the String \texttt{"null"}.

For every tuple received by the bolt, the \texttt{execute} method is called.  Using the sourceIPInt and destinationIPInt fields as input to the method \texttt{ipToAS} we determine the sourceAS and destinationAS fields that denote the source and destination AS number and name respectively. The new fields are appended to the received fields and all of them are emitted to the next bolt of the topology.

Algorithm \ref{algorithm:ip_to_as_bolt} outlines the IP to AS Bolt implementation. 

\begin{algorithm}[H]
\begin{algorithmic}[1]
\Function{prepare}{}
\State asMap = new TreeMap<Long, String[]>()
\For {line in ipToASFile}
\State fields = line.split(",")
\State asMap.put(fields[0], [fields[2], "start"])
\State asMap.put(fields[1], [fields[2], "stop"])
\EndFor
\EndFunction

\Function{ipToAS}{ipInt}
\State as = "null"
\State key = asMap.ceilingKey(ipInt)
\If {key != null}
\State value = asMap.get(key)
\If {(key == ipInt) || (value[1].equals("stop"))}
\State as = value[0]
\EndIf
\EndIf
\Return as
\EndFunction

\Function{execute}{tuple}
\State sourceIPInt = tuple.getField("sourceIPInt")
\State destinationIPInt = tuple.getField("destinationIPInt")
\State outputValues = tuple.getValues()
\State outputValues.add(ipToAS(sourceIPInt))
\State outputValues.add(ipToAS(destinationIPInt))
\State collector.emit(outputValues)
\EndFunction
\end{algorithmic}
\caption{IP to AS Bolt}
\label{algorithm:ip_to_as_bolt}
\end{algorithm}

\subsection{IP to DNS Bolt}

The natural join of the data stream and the Reverse DNS dataset is performed by the IP to DNS Bolt. The Reverse DNS dataset maps IP addresses to domain names. The data contained in the dataset have 2 fields: the IP address in dot-decimal notation and the corresponding domain name. Further information on the dataset is available in Subsection \ref{subsection:benchmarks_dns_dataset}.

The defining characteristic of the Reverse DNS dataset is that its size (55 GB uncompressed) is larger than the memory size, therefore loading it in every bolt's memory is not an option. To make the dataset available to the bolts, we store it in the \texttt{rnds} HBase table, where the IP addresses are used as the row key and the domain names are stored in the column \texttt{d:dns}. This allows the bolts to perform Get operations on the table for an IP address row key to get the corresponding domain name.

HBase can perform low latency Get operations by using \emph{Bloom filters}. A Bloom filter, is a data structure which is designed to predict whether a given element is a member of a set of data. A positive result from a Bloom filter is not always accurate, but a negative result is guaranteed to be accurate. In HBase, Bloom filters a lightweight in-memory structure to reduce the number of disk reads for a given Get operation to only the HFiles likely to contain the desired row.

The helper method \texttt{ipToDNS} takes an IP address in dot-decimal notation as input and returns corresponding domain name. More specifically, a Get operation is performed on the \texttt{rdns} HBase table for the input IP address row key. If the Get is successful, the corresponding domain name is the value of the column \texttt{d:dns} of the returned row, and is afterwards returned by the method. Otherwise the IP address provided does not have a corresponding domain name according to the dataset and the \texttt{ipToDNS} method returns the String \texttt{"null"}.

For every tuple received by the bolt, the \texttt{execute} method is called.  Using the sourceIP and destinationIP fields as input to the method \texttt{ipToDNS} we determine the sourceDNS and destinationDNS fields that denote the source and destination domain names respectively. The new fields are appended to the received fields and all of them are emitted to the next bolt of the topology.

Algorithm \ref{algorithm:ip_to_dns_bolt} outlines the IP to DNS Bolt implementation. 

\begin{algorithm}[H]
\begin{algorithmic}[1]
\Function{ipToDNS}{ip}
\State table = new HTable("rdns")
\State g = new Get(ip)
\State res = table.get(g)
\State dns = res.getValue("d", "dns")
\If {dns == null}
\State dns = "null"
\EndIf
\Return dns
\EndFunction

\Function{execute}{tuple}
\State sourceIP = tuple.getField("sourceIP")
\State destinationIP = tuple.getField("destinationIP")
\State outputValues = tuple.getValues()
\State outputValues.add(ipToDNS(sourceIP))
\State outputValues.add(ipToDNS(destinationIP))
\State collector.emit(outputValues)
\EndFunction
\end{algorithmic}
\caption{IP to DNS Bolt}
\label{algorithm:ip_to_dns_bolt}
\end{algorithm}

\subsection{Phoenix Bolt}

The last component of the topology is the Phoenix Bolt, which inserts the augmented data stream into the \texttt{netdata} Phoenix table. The table is described in detail in Section \ref{section:system_phoenix_table}. This bolt uses the Phoenix JDBC driver and performs an \texttt{UPSERT VALUES} query that includes all the fields received by the bolt. \texttt{UPSERT} queries are the only way to insert data in a table in Phoenix. This query inserts the row if not present, otherwise it updates the row's values in the table. In our case where the primary key of the table is the packet timestamp which is monotonically increasing this query behaves like \texttt{INSERT VALUES}.

\begin{lstlisting}[language=PhoenixSQL]
UPSERT INTO netdata VALUES (dateTime, sourceIP, sourceIPInt, destinationIP, destinationIPInt, protocol, sourcePort, destinationPort, ipSize, sourceAS, destinationAS, sourceDNS, destinationDNS);
\end{lstlisting}


\section{Phoenix Table}\label{section:system_phoenix_table}

After being processed by the Storm topology, the augmented data stream is stored at the \texttt{netdata} Phoenix table in HBase. The design of this table is important because it affects the way queries are executed. In our use case, the queries performed will be topN AS or topN DNS queries over a time window for the data. 

The queries performed on the table have a time window constraint. To benefit from HBase Scan operations that perform sequential reads, we want to use the packet's capture timestamp as the row key in the underlying HBase table. In this way, the table is sorted by capture timestamp and the data for any time window are stored sequentially. To achieve this we use the packet's capture timestamp as the primary key of the Phoenix table. The capture timestamp in microseconds can be used as the primary key since it is unique for each packet.

The use case queries concern either AS or DNS information. In HBase only the column families needed for the query are cached. Having separate column families containing AS, DNS and other information reduces query latency by reducing the data that have to be cached during each query. Therefore we separate the table's columns in 3 column families: one for the AS fields, another for DNS fields and a default column family that contains the rest of the packet's fields.

In HBase every cell value is always (when stored, transfered or cached) accompanied by its row key, column name and timestamp. Since the table will store millions of cells the column names will be repeates several millions of times in our data. This means that if the column names are large then the table size will be significantly increased. This is why we try to minimize the column names by keeping the column family and column qualifier names as small as possible.

Another way to reduce the table size is by utilizing Phoenix data types. Using the appropriate data type for each column reduces the size of each row, which improves query performance. For example, instead of storing the capture timestamp as string, we use the \texttt{BIGINT} type. The current UNIX timestamp in microseconds has 16 digits. As a string this needs 16 bytes to be stored, whereas a \texttt{BIGINT} needs only 8 bytes.

Having all the aforementioned design choices taken into consideration we create the \texttt{netdata} Phoenix table with the columns listed below. The dots in the column names separate the column families from the column qualifiers created in the underlining HBase table.
\begin{itemize}
\item \texttt{t}: Unix timestamp of the packet's capture time in microseconds, used as the primary key
\item \texttt{d.ipS}: source IP address in dot-decimal notation
\item \texttt{d.ipSI}: integer representation of the source IP address
\item \texttt{d.ipD}: destination IP address in dot-decimal notation
\item \texttt{d.ipDI}: integer representation of the destination IP address
\item \texttt{d.proto}: IP protocol number of the packet
\item \texttt{d.portS}: source port number
\item \texttt{d.portD}: destination port number
\item \texttt{d.size}: total length of the IP packet
\item \texttt{as.asS}: AS number and name of the source IP address
\item \texttt{as.asD}: AS number and name of the destination IP address
\item \texttt{dns.dnsS}: domain name of the source IP address
\item \texttt{dns.dnsD}: domain name of the destination IP address
\end{itemize}

The final table creation query, including the optimizations that will be described in Chapter \ref{chapter:optimizations} is presented below.

\begin{lstlisting}[language=PhoenixSQL]
CREATE TABLE netdata (
    t BIGINT PRIMARY KEY,
    d.ipS VARCHAR,
    d.ipSI BIGINT,
    d.ipD VARCHAR,
    d.ipDI BIGINT,
    d.proto SMALLINT,
    d.portS INTEGER,
    d.portD INTEGER,
    d.size INTEGER,
    as.asS VARCHAR,
    as.asD VARCHAR,
    dns.dnsS VARCHAR,
    dns.dnsD VARCHAR
) 
SALT_BUCKETS = 4,
DEFAULT_COLUMN_FAMILY = 'd',
DATA_BLOCK_ENCODING = 'NONE',
COMPRESSION = 'SNAPPY';
\end{lstlisting}


\cleardoublepage
